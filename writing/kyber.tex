
Let

\begin{equation}\label{polyn-quot-256}
	R_q := F[X]/\la X^{256} + 1 \ra
\end{equation}

where $F := \ZZ_{7681}$ be the ring of coefficients. Fast multiplication in
$R_q$ can be performed via the discrete Fourier transform (DFT) in the
following way: let $\mathbf{f}$ and $\mathbf{g}$ be two elements of $R_q$.
Then, their product is given by

\begin{eqnarray}\label{mult-Rq}
	\mathbf{f} \cdot \mathbf{g} = \left( \sum_{i = 0}^{255} f_i X^i
	\right) \cdot \left( \sum_{j = 0}^{255} g_j X^j \right) = \\
	\sum_{t = 0}^{255} \left( \sum_{i + j = t} f_i g_j \right) X^t,
	\nonumber 
\end{eqnarray}

where the operations on indices are modulo 256 and the operations on
coefficients $f_i, g_j$ are modulo 7681.

\bigskip

If we define the sequence $(h_t)$ by

\begin{equation}\label{h-seq-def}
	h_t := \sum_{i + j = t} f_i g_j
\end{equation}

then it is the \emph{discrete convolution}\footnote{The name is justified
by comparison with the classical continuous version} of the sequences
$(f_t)$ and $(g_t)$. Therefore, as proved in section 2 of
\cite{pollard-fftFiniteFields-1971}, one may compute their
product\footnote{More precisely, the right-hand side of equation
\eqref{dft-prod} is the sequence of coeficients of the product polynomial}
as

\begin{equation}\label{dft-prod}
	\mathbf{f} \cdot \mathbf{g} =
	\mathrm{DFT}^{-1} (\mathrm{DFT} (\mathbf{f}) \circ
	\mathrm{DFT}(\mathbf{g})),
\end{equation}

where DFT denotes the discrete Fourier transform\footnote{Also called
\emph{number-theoretic transform} and denoted NTT in the cryptography
context.} over $\ZZ_{7681}$ and ``\begin{math} \circ \end{math}'' denotes
the pointwise product of sequences.
